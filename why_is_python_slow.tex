% Preamble
\documentclass[12pt, aspectration=169]{beamer}
\usetheme{Copenhagen}
\usecolortheme{beaver}

\setbeamertemplate{navigation symbols}{}

% Packages
\usepackage{amsmath}
\usepackage{listings}
\usepackage{xcolor}
\usepackage{blkarray}
\usepackage{graphicx}
\usepackage{hyperref}

\definecolor{codegreen}{rgb}{0,0.6,0}
\definecolor{codegray}{rgb}{0.5,0.5,0.5}
\definecolor{codepurple}{rgb}{0.58,0,0.82}
\definecolor{backcolour}{rgb}{0.95,0.95,0.92}

\lstdefinestyle{mystyle}{
    backgroundcolor=\color{backcolour},
    commentstyle=\color{codegreen},
    keywordstyle=\color{magenta},
    numberstyle=\tiny\color{codegray},
    stringstyle=\color{codepurple},
    basicstyle=\ttfamily\footnotesize,
    breakatwhitespace=false,
    breaklines=true,
    captionpos=b,
    keepspaces=true,
    numbers=left,
    numbersep=5pt,
    showspaces=false,
    showstringspaces=false,
    showtabs=false,
    tabsize=4
}

\lstset{style=mystyle}

\title{Introduction to Advanced Python:\\Why is Python slow?}
\author{Kai Striega}
\date{\today}

% Document
\begin{document}

    \maketitle

    \begin{frame}{Before I start}
        \begin{block}{Here be dragons}<2->
            This talk introduces advanced topics, quickly.
            The content is designed to stretch your understanding and to challenge you.
        \end{block}
        \begin{block}{Questions}<3->
            I am happy to take questions during the talk, feel free to ask if something doesn't make sense.
        \end{block}
    \end{frame}

    \begin{frame}{What we're going to do today}
        \begin{itemize}
            \item Answer the question: Why is Python slow?
            \item Compare Python to C
            \item Discuss when execution time matters
        \end{itemize}
    \end{frame}

    \begin{frame}{Why C?}
        \begin{columns}
            \column{0.55\textwidth}
            \begin{itemize}
                \item C is \textit{very} fast
                \item C is compiled
                \item CPython is written in C
                \item Many Python extensions are written in C
            \end{itemize}
            \column{0.45\textwidth}
            \includegraphics[scale=0.35]{static/images/390px-Curiosity_Self-Portrait_at_'Big_Sky'_Drilling_Site}
        \end{columns}
    \end{frame}

    \begin{frame}{What is a \textit{compiled} langauge}
        \begin{center}
            \includegraphics[scale=0.25]{static/images/how_python_works}
        \end{center}
    \end{frame}

    \begin{frame}{What the Central Processing Unit (CPU) does}
        \begin{itemize}
            \item The CPU is responsible for executing instructions provided by software programs
            \item It performs calculations, manages data movement, and coordinates the operations of various hardware components
            \item It can be thought of as the ``brain'' of the computer
        \end{itemize}
        \begin{center}
            \begin{columns}
                \column{0.35\textwidth}
                \includegraphics[scale=0.25]{static/images/12th_Gen_Core_i9_12900K_Hero_Close_Up}
                \column{0.55\textwidth}
                \includegraphics[scale=0.25]{static/images/900px-alder_lake_die_2}
            \end{columns}
        \end{center}
    \end{frame}

    \begin{frame}{What we are measuring}
        \begin{itemize}
            \item What does it mean for a programming language to be fast?
            \item[] \textbf{Execution Speed} is the amount of time taken for the program to complete a particular task
            \item Many other measures of performance exists, execution speed is the simplest to work with
            \item We are measuring how quickly the CPU can execute the instructions generated by that language
        \end{itemize}
    \end{frame}

    \begin{frame}{What limits the performance of the CPU?}
        \begin{description}
            \item[Memory Bound] How quickly we can get data to the CPU
            \item[CPU Bound] How quickly the CPU can execute instructions
        \end{description}
    \end{frame}

    \begin{frame}{Today's Example}
        \lstinputlisting[language=Python,label={lst:python}]{code/sum_of_first_n_numbers.py}
    \end{frame}

    \begin{frame}{Is Python slow?}
        \begin{itemize}
            \item[]<2-> \texttt{time python sum\_of\_first\_n\_numbers.py\newline}
            \item[]<3-> \texttt{total=18446744064889498501\newline
            real 5m56.581s\newline
            user 5m56.518s\newline
            sys 0m0.004s\newline}
        \end{itemize}
    \end{frame}

    \begin{frame}{Today's Example, but in C}
        \lstinputlisting[language=C,label={lst:C}]{code/sum_of_first_n_numbers.c}
    \end{frame}

    \begin{frame}{What about C?}
        \begin{itemize}
            \item[]<2-> \texttt{time bin/sum\_of\_first\_n\_numbers\_O3\newline}
            \item[]<3-> \texttt{total=18446744064889498501\newline
            real 0m0.000s\newline
            user 0m0.000s\newline
            sys 0m0.000s\newline}
        \end{itemize}
    \end{frame}

    \begin{frame}{Why?}
        \begin{itemize}
            \item \textit{0m0.000s} seems like a bug\ldots it's not!
            \item C is a \textbf{compiled} langauge
            \item The compiler can perform optimizations on the code
        \end{itemize}
    \end{frame}

    \begin{frame}{Optimising Compilers}
        \begin{itemize}
            \item Our problem is always going to have the same answer
            \item The compiler is able to notice this
            \item And calculates the result ahead of time
            \item So how do we get around this?
        \end{itemize}
    \end{frame}

    \begin{frame}{What if we don't use optimizations?}
        \begin{itemize}
            \item[]<2-> \texttt{time bin/sum\_of\_first\_n\_numbers\_arg\_O0\newline}
            \item[]<3-> \texttt{total=18446744064889498501\newline
            real 0m2.020s\newline
            user 0m2.010s\newline
            sys 0m0.010s\newline}
        \end{itemize}
    \end{frame}

    \begin{frame}{What we've done so far}
        \begin{itemize}
            \item We've turned a 6-minute problem into a 2-second problem
            \item By changing the language
            \item This speedup exists even without optimizations
        \end{itemize}
    \end{frame}

    \begin{frame}{Why?}
        \begin{center}
            \begin{columns}
                \column{0.65\textwidth}
                \begin{itemize}
                    \item Interpreted
                    \item Dynamically Typed
                    \item Global Interpreter Lock
                    \item Language Design
                    \item Flexibility
                    \item Memory Organisation
                \end{itemize}
                \column{0.35\textwidth}
                \includegraphics[scale=0.45]{static/images/big_question_mark}
            \end{columns}
        \end{center}
    \end{frame}

    \begin{frame}{PyObjects and memory access}
        \begin{itemize}
            \item In Python \textit{everything} is an object.
            \item In the source code, this is implemented as a \textit{PyObject}
            \item A \textit{PyObject} can be thought of as a wrapper around some data and associated metadata
        \end{itemize}
    \end{frame}

    \begin{frame}{Memory Latencies}
        \begin{center}
            \begin{tabular}{|c c|}
             \hline
              & Latency \\ [0.5ex]
             \hline\hline
             L1 Cache & $<$1ns \\
             \hline
             L2 Cache & 4ns \\
             \hline
             Main Memory & 100ns \\
             \hline
             SSD & 16 $\mu$s \\ [1ex]
             \hline
             Round Trip (Datacenter)   & 0.5ms     \\
             \hline
             Round Trip (US to Europe) & 150 ms \\ [1ex]
             \hline
            \end{tabular}
        \end{center}
    \end{frame}

    \begin{frame}{Memory Latencies}
        \begin{center}
            \begin{tabular}{|c c|}
             \hline
              & Latency \\ [0.5ex]
             \hline\hline
             L1 Cache & $<$1 second \\
             \hline
             L2 Cache & 4 seconds \\
             \hline
             Main Memory & 100 seconds \\
             \hline
             SSD & 4.5 hours \\ [1ex]
             \hline
             Round Trip (Datacenter)   & 5.8 days     \\
             \hline
             Round Trip (US to Europe) & 5 years \\ [1ex]
             \hline
            \end{tabular}
        \end{center}
    \end{frame}

    \begin{frame}{A Python int is more than a C int}
        \begin{itemize}
            \item[] \begin{center}
                  \includegraphics[scale=0.5]{static/images/cint_vs_pyint}
            \end{center}
        \end{itemize}
    \end{frame}

    \begin{frame}{This extends to more complex data structures}
        \begin{center}
              \includegraphics[scale=0.25]{static/images/array_vs_list}
        \end{center}
    \end{frame}

    \begin{frame}{Performance Considerations}
        \begin{itemize}
            \item Want to get data from the \textit{fastest} possible source
            \item Usually that will be a \textit{cache} or \textit{memory} (RAM)
            \item Want to avoid network access as much as possible
        \end{itemize}
    \end{frame}

    \begin{frame}{Python's slowness doesn't matter}
        \begin{itemize}
            \item The number of problems where Python is \textit{truly} the bottleneck is very small
            \item Developers are expensive, Python is quick to develop with
            \item Many problems can be solved by throwing more hardware at them (horizontally scalable)
            \item Many problems are bound by network time, not CPU time
        \end{itemize}
    \end{frame}

    \begin{frame}{Further Reading}
        \begin{itemize}
            \item \url{https://www.oreilly.com/library/view/high-performance-python/9781492055013/}
            \item \url{https://www.amazon.com.au/Architecture-Departments-Electrical-Engineering-University-dp-0128119055/dp/0128119055/ref=dp_ob_title_bk}
            \item \url{https://www.oreilly.com/library/view/designing-data-intensive-applications/9781491903063/}
        \end{itemize}
        \begin{center}
            \begin{columns}
                \column{0.30\textwidth}
                \includegraphics[scale=0.35]{static/images/HighPerformancePython2ndEdition}
                \column{0.30\textwidth}
                \includegraphics[scale=0.35]{static/images/designing_data_intensive_applications}
                \column{0.30\textwidth}
                \includegraphics[scale=0.075]{static/images/computer_architecture}
            \end{columns}
        \end{center}
    \end{frame}

\end{document}