% Preamble
\documentclass[14pt, aspectration=169]{beamer}
\usetheme{Copenhagen}
\usecolortheme{beaver}

\setbeamertemplate{navigation symbols}{}

% Packages
\usepackage{amsmath}
\usepackage{listings}
\usepackage{xcolor}
\usepackage{blkarray}

\definecolor{codegreen}{rgb}{0,0.6,0}
\definecolor{codegray}{rgb}{0.5,0.5,0.5}
\definecolor{codepurple}{rgb}{0.58,0,0.82}
\definecolor{backcolour}{rgb}{0.95,0.95,0.92}

\lstdefinestyle{mystyle}{
    backgroundcolor=\color{backcolour},
    commentstyle=\color{codegreen},
    keywordstyle=\color{magenta},
    numberstyle=\tiny\color{codegray},
    stringstyle=\color{codepurple},
    basicstyle=\ttfamily\footnotesize,
    breakatwhitespace=false,
    breaklines=true,
    captionpos=b,
    keepspaces=true,
    numbers=left,
    numbersep=5pt,
    showspaces=false,
    showstringspaces=false,
    showtabs=false,
    tabsize=4
}

\lstset{style=mystyle}

\title{Why is Python slow?}
\author{Kai Striega}
\date{\today}

% Document
\begin{document}

    \maketitle

    \begin{frame}{What we're going to do today}
        \begin{itemize}
            \item<2-> Answer the question: Why is Python slow?
            \item<3-> Compare Python to another language C
            \item<4-> Introduce some of the key ideas in performance analysis
            \item<5-> Discuss when does execution time matters\ldots and when it doesn't
        \end{itemize}
    \end{frame}

    \begin{frame}{Before I start}
        \begin{block}{Here be dragons}<2->
            This talk introduces advanced topics, quickly.
            The content is designed to stretch your understanding and to challenge you.
        \end{block}
        \begin{block}{Questions}<3->
            I am happy to take questions during the talk, feel free to ask if something doesn't make sense.
        \end{block}
    \end{frame}

    \begin{frame}{Today's Example}
        \lstinputlisting[language=Python,label={lst:python}]{code/sum_of_first_n_numbers.py}
    \end{frame}

    \begin{frame}{Is Python slow?}
        \begin{itemize}
            \item[]<2-> \texttt{time python sum\_of\_first\_n\_numbers.py\newline}
            \item[]<3-> \texttt{total=18446744064889498501\newline
            real 5m56.581s\newline
            user 5m56.518s\newline
            sys 0m0.004s\newline}
        \end{itemize}
    \end{frame}

    \begin{frame}{Today's Example, but in C}
        \lstinputlisting[language=C,label={lst:C}]{code/sum_of_first_n_numbers.c}
    \end{frame}

    \begin{frame}{What about C?}
        \begin{itemize}
            \item[]<2-> \texttt{time bin/sum\_of\_first\_n\_numbers\_O3\newline}
            \item[]<3-> \texttt{total=18446744064889498501\newline
            real 0m0.000s\newline
            user 0m0.000s\newline
            sys 0m0.000s\newline}
        \end{itemize}
    \end{frame}

    \begin{frame}{Why?}
        \begin{itemize}
            \item \textit{0m0.000s} seems like a bug\ldots but it's not!
            \item C is a \textbf{compiled} langauge
            \item That means the compiler can perform optimizations before the code is run
        \end{itemize}
    \end{frame}

    \begin{frame}{What C does}
        \lstinputlisting[label={lst:asm-O3-no-arg}]{code/sum_clang_17_O3.asm}
    \end{frame}

    \begin{frame}{Today's Example, in C, with arguments}
        \lstinputlisting[language=C,label={lst:C-args}]{code/sum_of_first_n_numbers_arg.c}
    \end{frame}

    \begin{frame}{What C does (with an argument)}
        \lstinputlisting[label={lst:asm-O3-arg}]{code/sum_arg_clang_17_O3.asm}
    \end{frame}

    \begin{frame}{What C does (without optimisations)}
        \lstinputlisting[label={lst:asm-O3-arg-O0}]{code/sum_arg_clang_17_O0.asm}
    \end{frame}



\end{document}